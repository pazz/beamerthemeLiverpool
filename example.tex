\documentclass[aspectratio=169]{beamer}
\usetheme{Liverpool}

\usepackage{lipsum}


\title{Presentation Title\\to go here}
\subtitle{A Long Subtitle}
\author{Ron Weasley}

% short institute displayed in top right corner of front page
\institute[Computer Science]{Liverpool}
% set to empty to hide
%\institute[]{Liverpool}
% otherwise defaults to the institute
\institute{Liverpool}


\begin{document}
%\usebeamercolor[fg]{normal text}
% Title page can coloured differently by setting
%\setbeamercolor*{title}{bg=Pink, fg=White}
%
% to change the image on the right use
% \setbeamertemplate{background}[image right][your-image-file.png]
%
% default background and titling can be recovered by setting
% \setbeamertemplate{background}[default]
% \setbeamertemplate{title page}[default]
\begin{frame}[uolcolours=Teal Green]
\maketitle
\end{frame}

\begin{frame}[t]
    \frametitle{Slide Title}
    \framesubtitle{Second Line}

    Content.
    \lipsum[1][1-2]

    \begin{columns}[T]
        \begin{column}{0.5\textwidth}
            \begin{itemize}
                \item \lipsum[1][2]
                \item \lipsum[2][2]
                \item \lipsum[3][2]
            \end{itemize}
        \end{column}

        \begin{column}{0.5\textwidth}
            \begin{enumerate}
                \item \lipsum[3][1]
                \item \lipsum[4][2]
                \item \lipsum[5][2]
            \end{enumerate}
        \end{column}
    \end{columns}
\end{frame}

\section{Blocks}
\begin{frame}
\frametitle{Blocks}
There is some \alert{alerted} and some \emph{emphasized} text here. This is surprisingly far from the title?
\begin{block}{Titled Block}
    \lipsum[1][1-2] 
\end{block}
\begin{alertblock}{Alerted Block}
    \lipsum[1][1-2] 
\end{alertblock}
\begin{example}[Muggle]
%The prime minister
\end{example}

\end{frame}


\section{Blocks}

\begin{frame}
\frametitle{Blocks}
\framesubtitle{Maths environments}

\begin{definition}[Muggle]
A person without Magic
\end{definition}

\begin{lemma}
    This is a Lemma
\end{lemma}
\begin{theorem}
No Muggle knows about the magical world
\end{theorem} 
\begin{proof}
    by simple induction.
\end{proof}

\end{frame}


%%%%%%%%%%%%
% COLOURS
% 
\section{Colours}

% This is just to simplyfy the next slides
\newcommand{\printcolour}[1]{{\colorbox{#1}{\mbox{\strut}}\textcolor{#1}{#1}}}
\newcommand{\printcolourbox}[2]{{\colorbox{#1}{\textcolor{#2}{#1}}}}

\begin{frame}[t,fragile]
    \frametitle{COLORS}
    \framesubtitle{Available colours}

    The following UOL branded colours are available
    for use e.g.\ via \verb|\textcolor{..}|.

    \medskip
    \begin{columns}[onlytextwidth]
        \begin{column}{0.25\textwidth}
\printcolour{Charcoal}
\printcolour{Grey}
\printcolour{Stone}
        \end{column}
        \begin{column}{0.25\textwidth}
\printcolour{Dark Green}
\printcolour{Teal Green}
\printcolour{Burgundy}
        \end{column}
        \begin{column}{0.25\textwidth}
\printcolour{Pink}
\printcolour{Orange}
\printcolour{Yellow}
        \end{column}
        \begin{column}{0.25\textwidth}
\printcolour{Green}
\printcolour{Sky Blue}
\printcolour{Blue}
        \end{column}
    \end{columns}
\end{frame}
    \begin{frame}[t,fragile]
    \frametitle{COLORS}
    \framesubtitle{Profiles}

    This theme also defines some colour profiles that you can use for standout frames.
    They set several colour elements to sensible color combinations.

    \begin{block}{Defined Profiles}
        \printcolourbox{Sky Blue}{White}
        \printcolourbox{Teal Green}{White}
        \printcolourbox{Orange}{White}
        \printcolourbox{Grey}{White}
        \printcolourbox{Pink}{White}
        \printcolourbox{Burgundy}{White}
        \printcolourbox{Dark Green}{White}
        \printcolourbox{Blue}{White}
        %\colorbox{White}{\textcolor{Black}{default}}
    \end{block}


    \begin{block}{How to use them}
 You can switch to some predefined colour profiles mid presentation in two ways:
 \begin{enumerate}
     \item use \verb|\uolcolours{PRESET}| to cause subsequent frames to use the colours from PRESET;
     \item 
         Give a parameter "uolcolours" directly to a frame environment to affect only that one frame:
         \verb|\begin{frame}[uolcolours=PRESET] ... \end{frame}|
 \end{enumerate}
    \end{block}
\end{frame}


% \begin{frame}[t]
%     \frametitle{AVAILABLE COLOURS}
%     \framesubtitle{OLD PALETTE}

%     \begin{columns}
%         \begin{column}{0.3\textwidth}
% \printcolour{Atlantic Blue}
% \printcolour{Presentation Blue}
% \printcolour{Sunrise Gold}
% \printcolour{Classic Black}
% \printcolour{White}
% \printcolour{Deep plum}
% \printcolour{Violet}
%         \end{column}
%         \begin{column}{0.3\textwidth}
% \printcolour{Indigo}
% \printcolour{Cyan}
% \printcolour{Regal blue}
% \printcolour{Turquoise}
% \printcolour{Rasberry}
% \printcolour{Rose}
% \printcolour{Berry}
%         \end{column}
%         \begin{column}{0.3\textwidth}
% \printcolour{Tangerine}
% \printcolour{Pumpkin}
% \printcolour{Lemon}
% \printcolour{Lime}
% \printcolour{Leaf green}
% \printcolour{Pepper green}
%         \end{column}
%     \end{columns}
% \end{frame}

\uolcolours{Teal Green}
\begin{frame}
    \uolbrick[font=\LARGE\bfseries]{HELLO}
    \begin{block}{Here is some text in a block}
      \lipsum[1][1-5]
    \end{block}
\end{frame}

\begin{frame}[uolcolours=Orange]
    \uolbrick[font=\LARGE\bfseries]{THERE}
    \begin{block}{Here is some more text in a block}
        \lipsum[1][1-5]\footnote{This is a slide without tile/subtitle but with a custom UOL brick and a titled block.}
    \end{block}
\end{frame}

\begin{frame}
\frametitle{Rules}
\framesubtitle{Rules of Magical People}


The following rules apply to all witches and wizards:
\begin{itemize}
\item Don't eat Muggles
\item Don't kill Muggles
\end{itemize}

\end{frame}



%%%%%%%%%%%%%%%%%%%%%%%%%%%%%%%%%%5
% COLOUR PROFILES
% can be enabled by passing the uolcolours argument to a frame environment.
% e.g. \begin{frame}[t,uolcolours={Dark Green}]
%
% The available keys are
% Teal Green
% Skye Blue
% Orange
% Grey
% Pink
% Burgundy
% Dark Green
% Burgundy


\newcommand{\democontent}{
    Content.
    \lipsum[1][1-2]

    \begin{columns}[T]
        \begin{column}{0.5\textwidth}
            \begin{itemize}
                \item \lipsum[1][2]
                \item \lipsum[2][2]
                \item \lipsum[3][2]
            \end{itemize}
        \end{column}

        \begin{column}{0.5\textwidth}
            \begin{enumerate}
                \item \lipsum[3][1]
                \item \lipsum[4][2]
                \item \lipsum[5][2]
            \end{enumerate}
        \end{column}
    \end{columns}
}
%\pgfkeys{/uol/color/style=Pink}

\begin{frame}%[uolcolours]
    \frametitle{COLOUR PROFILES}
    \framesubtitle{DARK GREEN}

    \democontent
\end{frame}
\begin{frame}%[uolcolours=Pink]
    \frametitle{COLOUR PROFILES}
    \framesubtitle{Pink}
    \democontent
\end{frame}
\uolcolours{Dark Green}
\begin{frame}[uolcolours=Teal Green]
    \frametitle{COLOUR PROFILES}
    \framesubtitle{DARK GREEN}
    \democontent
    %\pgfkeysvalueof{/uol/color/styleprevious}
\end{frame}
\begin{frame}
    \frametitle{COLOUR PROFILES}
    \framesubtitle{DEFAULT}
    \democontent
\end{frame}

\end{document}
