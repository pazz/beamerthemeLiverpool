\documentclass[
aspectratio=169,
]{beamer}
\usetheme{Liverpool}

\title{Presentation Title\\to go here}
\subtitle{A Long Subtitle}
\author{Ron Weasley}
\institute{Computer Science}


\begin{document}

% Background: add an image to the right as follows.
%   \setbeamertemplate{background}[image right][YOURIMAGE]
% The default background (colour only) can be recovered by setting
%   \setbeamertemplate{background}[default]

% Titling: this theme redefines the positions of title, author, date and institute.
% The default titling can be recovered by setting
%   \setbeamertemplate{title page}[default]
% To set fancy colours use the \uolcolours command or frame-parameter as described below.


% TITLE DEMO 1
% Let's fence the title frame so that theming is local.
{
\setbeamertemplate{background}[image right][extra/uol-vg.jpg]
\uolcolours{Blue}

\begin{frame}
\maketitle
\end{frame}
}

% TITLE DEMO 2
\setbeamertemplate{background}[image right][extra/uol-vg-bw.png]  % this is document-wide
\begin{frame}[uolcolours=Pink]  % use uolcolours parameter to make frame follow colour profile
\maketitle
\end{frame}
\setbeamertemplate{background}[default]  % reset the background


%%%%%%%%%%%%%%%%%%%%%%%%%%%%%%%%%%%%%%%%%%%%%%%%%%%%%%%%%%%%%%%%%%
% DEMO CONTENT FOR THE EXAMPLE PRESENTATION 
\newcommand{\democontent}{
    
    This is some text content with and \emph{emphasis}, some \textbf{boldface} stuff and also
    \alert{alerted text}. The slide also contains footnotes\footnote{don't use me lots} and common beamer contstructs.

    \begin{columns}[T]
        \begin{column}{0.5\textwidth}
            \begin{itemize}
                \item Here is an item
                \item and another one.
                    \begin{itemize}
                        \item subitems look
                        \item like this
                    \end{itemize}
                \item another important point
            \end{itemize}
            \begin{enumerate}
                \item Enumerations look like this
                \item Humpty Dumpty
                    \begin{enumerate}
                        \item sat on a wall
                        \item had a great fall
                    \end{enumerate}
            \end{enumerate}
        \end{column}

        \begin{column}{0.5\textwidth}
            \begin{block}{A Titled Block}
                With some text content
            \end{block}
            \begin{alertblock}{Alert Block}
                With some important text.
            \end{alertblock}
            \begin{exampleblock}{Example Block}
                with example content.
            \end{exampleblock}
            

        \end{column}
    \end{columns}
}
%%%%%%%%%%%%%%%%%%%%%%%%%%%%%%%%%%%%%%%%%%%%%%%%%%%%%%%%%%%%%%%%%%

\begin{frame}[t]
    \frametitle{Demo Frame}
    \framesubtitle{that showcases how content looks}

    \democontent 
\end{frame}

\section{Blocks}

\begin{frame}
    \frametitle{Maths environments}

\begin{definition}[Muggle]
A person without Magic
\end{definition}

\begin{lemma}
    This is a Lemma
\end{lemma}
\begin{theorem}
No Muggle knows about the magical world
\end{theorem} 
\begin{proof}
    by simple induction.
\end{proof}

\end{frame}

%%%%%%%%%%%%%%%%%%
% BRICKS
%

\section{Bricks}
\begin{frame}[fragile]
    \frametitle{BRICKS}
    You can insert UOL-style bricks via the \verb|\uolbrick| and \verb|\uolbrick| commands.\\
    (They are defined in \emph{uolbricks.sty} and can be used outside of this beamer theme.)

    \bigskip
    \uolbricks{Here is a pair}{of bricks}
    and here
    \uolbrick[]{is a single brick}

    \medskip
    {
    \tiny
    \uolbricks{They shrink}{with the font size} 
    }
    {
    \Huge
    \uolbrick{OR GROW} 
    }
    \uolbricks[draw=black,fill=Pink,text=White][draw=Blue,fill=Sky Blue,text=Blue]{and can be themed}{individually} 

    \medskip
    \uolbricks[][][south west][north west]{You can arrange them}{like this}
    \uolbricks[][][south east][north east]{or like}{this}
    \uolbricks[][][east][west]{or like}{this}
    \uolbricks[][][north east][south]{or like}{this}
    \uolbricks[][][south west][north]{or like}{this}

    \vfill
\end{frame}

%%%%%%%%%%%%
% COLOURS
% 
\section{Colours}

% This is just to simplyfy the next slides
\newcommand{\printcolour}[1]{{\colorbox{#1}{\mbox{\strut}}\textcolor{#1}{#1}}\\}
\newcommand{\printcolourbox}[3]{{\scriptsize\uolbrick[fill=#1,text=#2,draw=black]{#3}}}



\begin{frame}[t,fragile]
    \frametitle{COLOURS}
    \framesubtitle{Available colours}

    The following UOL branded colours are available
    for use e.g.\ via \verb|\textcolor{..}|.

    \medskip
    \begin{columns}[onlytextwidth]
        \begin{column}{0.25\textwidth}
\printcolour{Charcoal}
\printcolour{Grey}
\printcolour{Stone}
        \end{column}
        \begin{column}{0.25\textwidth}
\printcolour{Dark Green}
\printcolour{Teal Green}
\printcolour{Burgundy}
        \end{column}
        \begin{column}{0.25\textwidth}
\printcolour{Pink}
\printcolour{Orange}
\printcolour{Yellow}
        \end{column}
        \begin{column}{0.25\textwidth}
\printcolour{Green}
\printcolour{Sky Blue}
\printcolour{Blue}
        \end{column}
    \end{columns}
\end{frame}
% \begin{frame}[t]
%     \frametitle{AVAILABLE COLOURS}
%     \framesubtitle{OLD PALETTE}

%     \begin{columns}
%         \begin{column}{0.3\textwidth}
% \printcolour{Atlantic Blue}
% \printcolour{Presentation Blue}
% \printcolour{Sunrise Gold}
% \printcolour{Classic Black}
% \printcolour{White}
% \printcolour{Deep plum}
% \printcolour{Violet}
%         \end{column}
%         \begin{column}{0.3\textwidth}
% \printcolour{Indigo}
% \printcolour{Cyan}
% \printcolour{Regal blue}
% \printcolour{Turquoise}
% \printcolour{Rasberry}
% \printcolour{Rose}
% \printcolour{Berry}
%         \end{column}
%         \begin{column}{0.3\textwidth}
% \printcolour{Tangerine}
% \printcolour{Pumpkin}
% \printcolour{Lemon}
% \printcolour{Lime}
% \printcolour{Leaf green}
% \printcolour{Pepper green}
%         \end{column}
%     \end{columns}
% \end{frame}

\begin{frame}[t,fragile]
    \frametitle{COLOURS}
    \framesubtitle{Colour profiles}

    This theme also defines colour profiles that 
    set several elements to sensible colour combinations.
    They are named after the dominant background colour:

        \printcolourbox{Sky Blue}{White}{Sky Blue}
        \printcolourbox{Teal Green}{White}{Teal Green}
        \printcolourbox{Orange}{White}{Orange}
        \printcolourbox{Grey}{White}{Grey}
        \printcolourbox{Pink}{White}{Pink}
        \printcolourbox{Burgundy}{White}{Burgundy}
        \printcolourbox{Dark Green}{White}{Dark Green}
        \printcolourbox{Blue}{White}{Blue}
        \printcolourbox{White}{black}{default}

        \vfill
        You can switch to some predefined colour profiles mid presentation in two ways:
 \begin{enumerate}
     \item use \verb|\uolcolours{PRESET}| to cause subsequent frames to use the colours from PRESET;
     \item 
         Give a parameter "uolcolours" directly to a frame environment to affect only that one frame:
         \verb|\begin{frame}[uolcolours=PRESET] ... \end{frame}|
 \end{enumerate}
\end{frame}



\begin{frame}[uolcolours=Sky Blue]
    \uolbrick[font=\LARGE\bfseries]{HELLO}
    \begin{block}{Here is some text in a block}
      Ut purus elit, vestibulum ut, placerat ac, adipiscing vitae, felis. Curabitur dictum
gravida mauris. Nam arcu libero, nonummy eget, consectetuer id, vulputate a, magna.
Donec vehicula augue eu neque. Pellentesque habitant morbi tristique senectus et
netus et malesuada fames ac turpis egestas. Mauris ut leo.
    \end{block}
\end{frame}
\begin{frame}[uolcolours=Sky Blue]
    \frametitle{COLOUR PROFILE}
    \framesubtitle{SKY BLUE}
    \democontent
\end{frame}

\begin{frame}[uolcolours=Teal Green]
    \uolbrick[font=\LARGE\bfseries]{HELLO!}
    \begin{block}{Here is some more text in a block}
        Ut purus elit, vestibulum ut, placerat ac, adipiscing vitae, felis. Curabitur dictum
gravida mauris. Nam arcu libero, nonummy eget, consectetuer id, vulputate a, magna.
Donec vehicula augue eu neque. Pellentesque habitant morbi tristique senectus et
netus et malesuada fames ac turpis egestas.
        Mauris ut leo\footnote{This is a slide without tile/subtitle but with a custom UOL brick and a titled block.}.
    \end{block}
\end{frame}
\begin{frame}[uolcolours=Teal Green]
    \frametitle{COLOUR PROFILE}
    \framesubtitle{TEAL GREEN}
    \democontent
\end{frame}

\uolcolours{Orange}
\begin{frame}
    \uolbrick[font=\LARGE\bfseries]{HELLO ORANGE!}
    \begin{block}{Here is some more text in a block}
        Ut purus elit, vestibulum ut, placerat ac, adipiscing vitae, felis. Curabitur dictum
gravida mauris. Nam arcu libero, nonummy eget, consectetuer id, vulputate a, magna.
Donec vehicula augue eu neque. Pellentesque habitant morbi tristique senectus et
netus et malesuada fames ac turpis egestas. Mauris ut leo.
    \end{block}
\end{frame}

\begin{frame}
    \frametitle{COLOUR PROFILE}
    \framesubtitle{ORANGE}
    \democontent
\end{frame}


\begin{frame}[uolcolours=Pink]
    \frametitle{COLOUR PROFILE}
    \framesubtitle{PINK}
    \democontent
\end{frame}

\begin{frame}[uolcolours=Grey]
    \frametitle{COLOUR PROFILE}
    \framesubtitle{GREY}
    \democontent
\end{frame}

\begin{frame}[uolcolours=Burgundy]
    \frametitle{COLOUR PROFILE}
    \framesubtitle{BURGUNDY}
    \democontent
\end{frame}

\begin{frame}[uolcolours=Dark Green]
    \frametitle{COLOUR PROFILE}
    \framesubtitle{DARK GREEN}
    \democontent
\end{frame}

\begin{frame}[uolcolours=Blue]
    \frametitle{COLOUR PROFILE}
    \framesubtitle{BLUE}
    \democontent
\end{frame}

\uolcolours{default}
\begin{frame}
    \frametitle{COLOUR PROFILE}
    \framesubtitle{DEFAULT}
    \democontent
\end{frame}

\end{document}
