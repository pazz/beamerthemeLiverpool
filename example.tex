\documentclass[aspectratio=169]{beamer}
\usetheme{Liverpool}

\usepackage{lipsum}


\title{Presentation Title\\to go here}
\subtitle{A Long Subtitle}
\author{Ron Weasley}

% short institute displayed in top right corner of front page
\institute[Computer Science]{Liverpool}
% set to empty to hide
%\institute[]{Liverpool}
% otherwise defaults to the institute
%\institute{Liverpool}



\begin{document}

\begin{frame}
\maketitle
\end{frame}


\begin{frame}[t]
    \frametitle{Slide Title}
    \framesubtitle{Second Line}

    Content
    \begin{itemize}
        \item Bullet point
    \end{itemize}
\end{frame}


%\setbeamercolor{normal text}{bg=Indigo,fg=White}
% \setbeamercolor{structure}{fg=White}
 
\begin{frame}
\frametitle{Muggles}
We first explore what a Muggle is 

\begin{definition}[Muggle]
A person without Magic
\end{definition}

\begin{theorem}
No Muggle knows about the magical world
\end{theorem} 

\begin{example}[Muggle]
The prime minister
\end{example}

\end{frame}

%\setbeamercolor{normal text}{bg=Tangerine,fg=White}
% \setbeamercolor{normal text}{bg=Indigo,fg=White}
% \setbeamercolor{structure}{fg=White}

\begin{frame}[t]

    \vspace{2cm}
    {\LARGE\uolbrick{HELLO}}
    \begin{block}{Here is some text in a block}
        \lipsum[1][1-5]
    \end{block}
    
\end{frame}


\begin{frame}
\frametitle{Rules}
\framesubtitle{Rules of Magical People}

The following rules apply to all witches and wizards:
\begin{itemize}
\item Don't eat Muggles
\item Don't kill Muggles
\end{itemize}

\end{frame}


\newcommand{\printcolour}[1]{
    {\colorbox{#1}{\mbox{\strut}}\textcolor{#1}{#1}\\}
}
\setbeamercolor{normal text}{bg=white,fg=black}
\begin{frame}[t]
    \frametitle{AVAILABLE}
    \framesubtitle{COLORS}

    \begin{columns}
        \begin{column}{0.3\textwidth}
\printcolour{Atlantic Blue}
\printcolour{Presentation Blue}
\printcolour{Sunrise Gold}
\printcolour{Classic Black}
\printcolour{White}
\printcolour{Deep plum}
\printcolour{Violet}
        \end{column}
        \begin{column}{0.3\textwidth}
\printcolour{Indigo}
\printcolour{Cyan}
\printcolour{Regal blue}
\printcolour{Turquoise}
\printcolour{Rasberry}
\printcolour{Rose}
\printcolour{Berry}
        \end{column}
        \begin{column}{0.3\textwidth}
\printcolour{Tangerine}
\printcolour{Pumpkin}
\printcolour{Lemon}
\printcolour{Lime}
\printcolour{Leaf green}
\printcolour{Pepper green}
        \end{column}
    \end{columns}
\end{frame}

 
\end{document}
